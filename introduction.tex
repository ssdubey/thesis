When applications replicate data across different sites, they need to make a
fundamental choice regarding the consistency of data. Strong consistency properties such
as Linearizability~\cite{Herlihy90} and Serializability~\cite{Bernstein79}
makes it easier to design correct applications. However, strong consistency is often at
odds with high performance. Strong consistency necessitates that all the replicas
coordinate to agree on a global order in which any conflicting operations are
resolved. The CAP theorem~\cite{Gilbert02} and PACELC theorem~\cite{Abadi12}
state that strongly consistent applications exhibit higher latencies when the all
the replicas are reachable, and they are unavailable when some of the replicas are
unreachable. This limitation has spurred the development of commercial weakly
consistent distributed databases for wide-area applications such as
DynamoDB~\cite{DynamoDB}, Cassandra~\cite{Cassandra}, CosmosDB~\cite{CosmosDB}
and Riak~\cite{Riak}. However, developing correct applications under weak
consistency is challenging due to the fact that the operations may be reordered
in complex ways even if issued by the same session~\cite{Burckhardt14}.
Moreover, these databases only offer a limited set of sequential data types
with a built-in conflict resolution strategies such as last-write-wins and
multi-valued objects. Such built-in conflict resolution leads to anomalies such
as write-skew~\cite{Berenson95} which makes it difficult (and often impossible)
to develop complex applications with rich behaviours.

Rather than programming with sequential data types while reasoning about their
semantics in a weakly consistent setting, an alternative strategy is to equip
the data types with the ability to reconcile conflicts. Kaki et
al.~\cite{Kaki19} recently proposed Mergeable Replicated Data Types (MRDTs) as
a way to automatically derive correct distributed variants of ordinary data
types. The inductively defined data types are equipped with an invertible
relational specification which is used to derive a three-way merge function \`a
la Git~\cite{Git}, a distributed version control system.

What does it take to make MRDTs a practical alternative to implementing
high-throughput, low-latency distributed applications such as the ones that
would be implemented over industrial-strength distributed databases? There are
several key challenges to getting there.

Firstly, while MRDTs define merge semantics for
operations on individual objects, Kaki et al. do not describe the semantics of
composition of operations on multiple objects -- i.e. transactions. Transactions
are indispensable for building complex applications. Strongly consistent
distributed transactions suffer from unavailability~\cite{Abadi12}, whereas
highly-available transactions~\cite{Bailis13} combined with weakly consistent
operations often lead to incomprehensible behaviours~\cite{Viotti16}.

Secondly, MRDTs impose significant burden on the storage and network layer
to be able to support three-way merges to reconcile conflicts. Kaki et al.
implement MRDTs over Irmin~\cite{Irmin}, a Git-like store for arbitrary
objects, not just files. As with Git, in order to reconcile conflicts,
three-way merges in MRDTs require the storage layer to record enough history to
be able to retrieve the \emph{lowest common ancestor} (LCA) state. For a
distributed database, performance of the network layer is quite important for
throughput and latency. Industrial-strength distributed databases use gossip
protocols~\cite{Kermarrec07} to quickly disseminate updates in order to ensure
fast convergence between the replicas. Git comes equipped with a remote
protocol for transferring objects between remote sites using \emph{push} and
\emph{pull} mechanisms. Unfortunately, directly using the Git remote protocols
would mean that the client will have to name branches explicitly, complicating
the programming model. The onus is on the client to ensure that all the
branches that have updates are merged in order to ensure that there is
convergence. This is undesirable.

\paragraph*{\textbf{Contributions.}} In this paper, we present \name, a
programming model for building loosely connected distributed applications that provides
coordination-free transactions over MRDTs. \name provides per-object causal
consistency, and the transaction model is built on the principles of Git-like
branches. Rather than relying on Git remote protocol for dissemination across
replicas, we instantiate \name on top of Cassandra, an industrial-strength,
off-the-shelf distributed store~\cite{10.1145/1773912.1773922}. Unlike Git,
\name does not expose named branches explicitly, and ensures eventual
convergence. Importantly, \name only relies on eventual consistency, and \name
can be instantiated on any eventually consistent key-value store. Extensive
evaluation shows that \name makes it easy to build complex high-performance
distributed applications.

The rest of the paper is organised as follows. We motivate the \name model by
designing a distributed build cache in the next section.
Section~\ref{sec:model} describes the \name programming model.
Section~\ref{sec:impl} describes the instantiation of \name on Cassandra. We
evaluate the instantiation of \name on top of Cassandra in
section~\ref{sec:eval}. Sections~\ref{sec:related} and \ref{sec:conc} present
the related work and conclusions, respectively.
