Several prior works have addressed the challenge of balancing the
programmability and performance under eventual consistency. RedBlue
consistency~\cite{Li12} offers causal consistency by default (blue), but
operations that require strong consistency (red) are executed in single total
order. Quelea~\cite{Sivaramakrishnan15} and MixT~\cite{Milano18} offer
automated analysis for classifying and executing operations are different
consistency levels embedded in weakly isolated transactions, paying the cost of
proportional to the consistency level. Indeed, mixing weaker consistency and
transactions have been well-studied~\cite{Brutschy17,Kraska13,CosmosDB}.

\name only supports causal consistency, but it is known to be the strongest
consistency level that remains available~\cite{Llyod11}. While prior works
attempt to reconcile traditional isolation levels with weak consistency, \name
leaves the choice of reading and writing updates to and from other transactions
to the client through the use of |publish| and |refresh|. We believe that
traditional database isolation levels are already quite difficult to get
right~\cite{Kaki17}, and attempting to provide a fixed set of poorly understood
isolation levels under weak consistency will lead to proliferation of bugs.

\name is distinguished by the equipping data types with the ability to handle
conflicts (three-way merge functions). \name builds on top of
Irmin~\cite{Irmin} library. Irmin allows arbitrary branching and merging
between different branches at the cost of having to expose the branch name.
\name refreshes and publishes implicitly to the public branch at a repository,
which obviates the need for naming branches explicitly. Irmin does not include
a distribution and convergence layer; \name uses Cassandra for this purpose.
\name provides causal consistency and coordination free transactions over
weakly consistent Cassandra. Several prior work have similarly obtained
stronger guarantees weaker stores~\cite{Sivaramakrishnan15,Bailis13b}.

TARDiS~\cite{Crooks16} supports user-defined data types, and a transaction
model similar to \name. TARDiS is however a machine model that exposes the
details of explicit branches and merges to the developer, whereas \name is a
programming model that can be instantiated on any eventually consistent
key-value store. For instance, in TARDiS programmers need to invoke a separate
merge transaction that does an n-way merge. \name transaction model is more
flexible than TARDiS. For example, \name can support monotonic atomic view,
which TARDiS cannot -- TARDiS transactions do not have a way of allowing more
recent updates since the transaction began. TARDiS does not discuss merges
without LCAs or the issue with recursive merges. We found recursive merges to
be a very common occurrence in practice. Concurrent
revisions~\cite{Burckhardt12} describe a programming model with branch and
merge workflow with explicit branches and restrictions on the shape of history
graphs. \name makes the choice of branches to publish and refresh implicit
leading to a simpler model. Concurrent revisions does not include an
implementation.
